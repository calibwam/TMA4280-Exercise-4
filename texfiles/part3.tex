\section{OpenMPI implementation }
Making an OpenMPI implementation we used the restrictions given in the assignment, which allow us to make a cleaner implementation. 

\begin{enumerate}

\item We do a number of iterations which is a power of two. 

\item We use a number of processes which is a power of two. 

\end{enumerate}

\begin{align}
2^{m}/2^{n} = 2^{m-n}
\end{align}

One effect of this is that we can partition the array, $2^{m}$ in the equation above, by dividing the size by the number of processes, $2^{n}$ above, to achieve the size of each partition, $2^{m-n}$. 
Since all these numbers can be represented as integers this gives no loss when doing integer division. 

We split the work into a few steps, populating the array, sending out the arrays to the workers, adding the elements of the arrays, sending back the answer, receiving the answer and summing the sums together. 

The primary thread is the only thread generating the array, it then sends all the other threads a piece of the array. 
Every other thread waits to receive its part. 
At this point every thread got one partition of the array, so they all find the sum of the contents in their partition. 
Every thread except the primary thread then proceeds to send the data back to the primary thread, which have been reveiving data from the threads and makes the final sum. 


\section{OpenMP and OpenMPI }
The differnce between the OpenMPI implementation and the combination is that the primary thread is using OpenMP to split the job into several senders and using the same technique for receiving. 
In addition the workers use OpenMP to paralellize their work. 


